\documentclass[a4paper,11pt]{report}

\usepackage[utf8]{inputenc}
\usepackage[T1]{fontenc}
\usepackage[francais]{babel}
\usepackage{hyperref}
\hypersetup{pdfauthor={Julien Delplanque},
            pdftitle={Calcul des probabilités : Synthèse des formules},
            pdfsubject={probas},
            pdfkeywords={résumé}}
\usepackage{amsfonts}
\usepackage{amsmath}
\usepackage{listings}
\usepackage{fullpage}
\usepackage{fancyhdr}
\pagestyle{fancy}

\renewcommand{\thesection}{}
\renewcommand{\thesubsection}{}

\renewcommand{\headrulewidth}{0pt}
\fancyhead[C]{}
\fancyhead[L]{}
\fancyhead[R]{}

\renewcommand{\footrulewidth}{1pt}
\fancyfoot[C]{\textbf{\thepage}}
\fancyfoot[L]{\textsc{Delplanque} Julien}
\fancyfoot[R]{Version 2.1 - 2014-2015}

\begin{document}
\renewcommand{\labelitemi}{$\cdot$}
\renewcommand{\labelitemii}{$\diamond$}
\renewcommand{\labelitemiii}{$\cdot$}
\begin{Large}\begin{center}
   \underline{\textbf{Calcul des probabilités : Synthèse des formules}}
\end{center}\end{Large}

%%%%%
\section{Séance I : Introduction}
\begin{itemize}
	\item \textbf{Événement} : Un événement $A$ lié à une expérience aléatoire est un sous-ensemble des résultats possibles. Il est souvent défini par une proposition.
	\item \textbf{Probabilité d'un événement} : La probabilité d'un événement $A$ est définie comme $P(A)=$ nombre de cas favorables$/$nombre de cas possibles
	\item \textbf{Arbre pondéré} : Un arbre pondéré est un schéma permettant de résumer une expérience aléatoire en connaissant des probabilités conditionnelles.
	\item \textbf{Opérations ensemblistes :}
		\begin{itemize}
			\item \textbf{Événements contraires} : $A^C$ l'événement $A$ ne se produit pas.
			\item \textbf{Réalisation simultanée de deux événements} : $A \cap B$
			\item \textbf{Réalisation d'un événement au moins} : $A \cup B$
			\item \textbf{Événements indépendants} : $P(A \cap B) = P(A) P(B)$
			\item \textbf{Événements dépendants} : $P(A|B)$ (probabilité de B si A est arrivé)
			\item $A - B = A \cap B^C$
			\item $A \oplus B = A \cup B - A \cap B$
			\item $P(A \cup B) = P(A)+P(B)-P(A \cap B)$
			\item $P(A \cap B) = P(A)P_A(B)$
		\end{itemize}
	\item \textbf{Lois de De Morgan} :
		\begin{align*}
			\overline{A} \cap \overline{B} =& \overline{A \cup B} \\
			\overline{A} \cup \overline{B} =& \overline{A \cap B}
		\end{align*}
	\item \textbf{Espérance mathématique (ou moyenne)} :
		\[
			E(X)=\bar{X}=\sum\limits_{i=1}^{n}{x_ip_i}
		\]
	\item \textbf{Variance} :
		\[
			V(X)=\sum\limits_{i=1}^{n}{(x_i-\bar{X})^2p_i}
			\text{ avec }\bar{X} = E(X)
		\]
	\item \textbf{Écart-type} :
		\[
			\sigma(X) = \sqrt{V(X)}
		\]
\end{itemize}

%%%%%
\section{Séance II : Densité de probabilité}
\begin{itemize}
	\item \textbf{Densité de probabilité} : Une densité de probabilité est une fonction $f$ qui permet de représenter une loi de probabilité sous forme d'intégrale. Elle vérifie deux conditions :
		\[
			f(x) \ge 0 \quad \quad
			\int\limits_{-\infty}^{+\infty}{f(x)dx}=1
		\]
	\item \textbf{Fonction de répartition} : Une fonction de répartition $F_X(x)$ d'une variable aléatoire $X$ représente la loi de probabilité de cette variable aléatoire. Elle est définie comme :
		\[
			F_X(x)=\int\limits_{-\infty}^{x}{f(x)dx}
		\]
\end{itemize}

%%%%%
\section{Séance III : Loi de probabilité - Moments}
\begin{itemize}
	\item Une loi de probabilité est définie par :
		\[
			\forall k : P_X\{k\} \ge 0 \quad
			\sum\limits_{k=0}^{n}{P_X\{k\}} = 1
		\]
	\item \textbf{Loi continue ($\mathbb{R}$)} :
	\begin{itemize}
		\item \textbf{Moment d'ordre r} :
			\[
				\alpha_r(X) = \int\limits_{\mathbb{R}}{x^rf(x)dx}
			\]
		\item \textbf{Moment centré d'ordre r} :
			\begin{align*}
				\mu_r(X) =& \int\limits_{\mathbb{R}}{\left(x-\bar{X}\right)^rf(x)dx} \\
				\bar{X} =& \alpha_1 \\
				V=&\mu_2 \\
				\mu_1 =& 0 \\
				\mu_2 =& \sigma^2 = \alpha_2 - \alpha_1^2 \\
				\mu_3 =& \alpha_3 - 3\alpha_2 \alpha_1 + 2\alpha_1^3
			\end{align*}
	\end{itemize}
	\item \textbf{Moment factoriel d'ordre $r$ en loi discrète ($\mathbb{Z}$)} :
		\begin{align*}
			\alpha_{[r]}(X) =& \sum\limits_{i = 0}^{n}{\frac{x!}{(x-r)!}} \\
			P(x) =& E\left(\frac{X!}{(X-r)!}\right) \\
			\alpha_1 =& \alpha_{[1]} \\
			\alpha_2 =& \alpha_{[2]} + \alpha_{[1]} \\
			\alpha_3 =& \alpha_{[3]} + 3\alpha_{[2]} + \alpha_{[1]}
		\end{align*}
\end{itemize}

%%%%%
\section{Séance IV : Moments-Fonction caractéristique}
\begin{align*}
	\phi(t) =& \int\limits_{\mathbb{R}}{e^{itx}dp(x)} =
		\int\limits_{\mathbb{R}}e^{itx}f(x)dx \\
	|\phi(t)| \le& 1 \\
	\phi(0) =& 1 \\
	\phi(t) =& 1 + it\alpha_1 + \frac{(it)^2}{2!}\alpha_2 + \dots + \frac{(it)^r}{r!}\alpha_r \\
	\alpha_k =& \frac{\partial^k \phi(0)}{i^k}
\end{align*}

%%%%%
\section{Séance V : Sommes de variables aléatoires réelles}
\begin{itemize}
	\item Soient deux v.a.r. $X$ et $Y$ respectivement données par les densités de probabilité $f_X(k)$ et $f_Y(l)$ et soit $Z = X+Y \implies z = k+l$, on a alors par le produit de convolution que :
	\begin{itemize}
		\item Pour une densité continue :
			\[
				f_Z(Z) = \int\limits_{-\infty}^{+\infty}{f_X(k)f_Y(z-k)dk} = \int\limits_{-\infty}^{+\infty}{f_X(z-l)f_Y(l)dl}
			\]
		\item Pour une densité discrète :
			\[
				f_Z(Z) = \sum\limits_{k=-\infty}^{+\infty}{f_X(k)f_Y(z-k)} = \sum\limits_{l=-\infty}^{+\infty}{f_X(z-l)f_Y(l)}
			\]
	\end{itemize}
	\item Soit $f_X(x) = ae^{-ax}$ alors $\overline{x} = \alpha_1 = \frac{1}{\alpha}$ et $V = \mu_2 = \frac{1}{\alpha^2}$
\end{itemize}

%%%%%
\section{Séance VI : Théorème de la limite centrale et loi des grands nombres}
\begin{itemize}
	\item \textbf{Théorème de la limite centrale} \\
	Soit $X_i$, $i>1$ une suite de v.a.r. :
	\begin{itemize}
		\item indépendantes
		\item de même loi
		\item de variance finie $\sigma^2$
	\end{itemize}
	Alors,
	\[
		Z_n = \frac{\sum\limits_{i=0}^{n}{(x_i-A(X_i))}}{\sigma\sqrt{n}}
	\]
	est telle que
	\[
		P(Z_n \le x) \rightarrow
		\frac1{\sqrt{2\pi}}
		\int\limits_{-\infty}^{x}{e^{-y^2/2}dy}
		\text{ quand }n\rightarrow \infty
	\]
	\[
		Z_n \underset{n\to+\infty}{\longrightarrow} N(0,1)
	\]
	\item \textbf{Loi des grands nombres} :
	\begin{itemize}
		\item \textbf{Loi faible} :
		Soit $X_n$, $n \in \mathbb{N}$, une suite de v.a.r. indépendantes :
		\begin{itemize}
			\item soit de même loi et $E(X_1^2) < \infty$
			\item soit de lois quelconques et $\frac{1}{n^2}\sum\limits_{i}{\sigma^2(X_i) \rightarrow 0}$ (critère de Markov)
		\end{itemize}
		Alors :
		\[
			\lim\limits_{n\rightarrow \infty}
			{P\left(\left|\frac{1}{n}\sum\limits_{i}{X_i}-E(X_1)\right|
			>\varepsilon\right)} = 0
		\]
		Autrement dit, la moyenne arithmétique de $X_n$ converge vers l'espérance de $X_1$.
		\item \textbf{Loi forte} :\\
		Soit $X_n$, $n \in \mathbb{N}$, une suite de v.a.r. indépendantes :
		\begin{itemize}
			\item soit de même loi et $E(|X_1|) < \infty$ (critère de Kolmogorov)
			\item soit de lois quelconques et $\sum\limits_{i}{}\frac{1}{i^2}\sigma^2(X_i) < \infty$
		\end{itemize}
		Alors, presque pour toute réalisation $w$ :
		\[
			\lim\limits_{n \rightarrow \infty}
			\frac1n\sum\limits_i{X_i} = E(X_1)
		\]
	\end{itemize}
	On parle de convergence presque sûre.
	Autrement dit, la moyenne arithmétique de $X_n$ converge,
	pour presque chaque réalisation, vers l'espérance $X_1$.
\end{itemize}

%%%%%
\section{Formules utiles}
\begin{itemize}
\item \textbf{Intégrales spéciales} :
	\begin{align*}
		\int\limits_{-\infty}^{+\infty}{e^{-\alpha x^2}dx}=&
			\sqrt{\frac{\pi}{\alpha}} \\
			\int\limits_{-\infty}^{0}{x^ne^xdx}=&(-1)^nn!
		\end{align*}
\item \textbf{Symétrie et intégrales} :
	\begin{align*}
		f(-x) = -f(x) \rightarrow&
			\int\limits_{-\infty}^{+\infty}{f(x)dx}=0 \\
		f(-x) = f(x) \rightarrow&
			\int\limits_{-\infty}^{+\infty}{f(x)dx} =
			2\int\limits_{-\infty}^{0}{f(x)dx} =
			2\int\limits_{0}^{+\infty}{f(x)dx}
	\end{align*}
\item \textbf{Sommes} :
	\begin{align*}
		\sum\limits_{i=1}^{n}{i} =& \frac{n(n+1)}{2} \\
		\sum\limits_{i=1}^{n}{i^2} =& \frac{(2n+1)(n+1)n}{6} \\
		\sum\limits_{i=1}^{n}{i^3} =& \left(\sum\limits_{i=1}^{n}{i}\right)^2 \\
		\sum\limits_{i=1}^{n}{i^4} =& \frac{n}{30}(n+1)(2n+1)(3n^2+3n-1)
	\end{align*}
\item \textbf{La fonction gamma}, prolongement de la factorielle :
	\begin{align*}
		\forall n>0 \quad&
		\Gamma(n)=\int\limits_{0}^{+\infty}{t^{n-1}e^{-t}dt} \\
		&\Gamma(n+1)=n\Gamma(n) \\
		&\Gamma(1/2)=\sqrt{\pi} \\
		&\Gamma(n+1/2)=\frac{(2n)!}{2^{2n}n!}\sqrt{\pi}
	\end{align*}
\end{itemize}
\end{document}
